% Options for packages loaded elsewhere
\PassOptionsToPackage{unicode}{hyperref}
\PassOptionsToPackage{hyphens}{url}
%
\documentclass[
]{book}
\title{R Without Statistics}
\author{David Keyes}
\date{}

\usepackage{amsmath,amssymb}
\usepackage{lmodern}
\usepackage{iftex}
\ifPDFTeX
  \usepackage[T1]{fontenc}
  \usepackage[utf8]{inputenc}
  \usepackage{textcomp} % provide euro and other symbols
\else % if luatex or xetex
  \usepackage{unicode-math}
  \defaultfontfeatures{Scale=MatchLowercase}
  \defaultfontfeatures[\rmfamily]{Ligatures=TeX,Scale=1}
\fi
% Use upquote if available, for straight quotes in verbatim environments
\IfFileExists{upquote.sty}{\usepackage{upquote}}{}
\IfFileExists{microtype.sty}{% use microtype if available
  \usepackage[]{microtype}
  \UseMicrotypeSet[protrusion]{basicmath} % disable protrusion for tt fonts
}{}
\makeatletter
\@ifundefined{KOMAClassName}{% if non-KOMA class
  \IfFileExists{parskip.sty}{%
    \usepackage{parskip}
  }{% else
    \setlength{\parindent}{0pt}
    \setlength{\parskip}{6pt plus 2pt minus 1pt}}
}{% if KOMA class
  \KOMAoptions{parskip=half}}
\makeatother
\usepackage{xcolor}
\IfFileExists{xurl.sty}{\usepackage{xurl}}{} % add URL line breaks if available
\IfFileExists{bookmark.sty}{\usepackage{bookmark}}{\usepackage{hyperref}}
\hypersetup{
  pdftitle={R Without Statistics},
  pdfauthor={David Keyes},
  hidelinks,
  pdfcreator={LaTeX via pandoc}}
\urlstyle{same} % disable monospaced font for URLs
\usepackage{longtable,booktabs,array}
\usepackage{calc} % for calculating minipage widths
% Correct order of tables after \paragraph or \subparagraph
\usepackage{etoolbox}
\makeatletter
\patchcmd\longtable{\par}{\if@noskipsec\mbox{}\fi\par}{}{}
\makeatother
% Allow footnotes in longtable head/foot
\IfFileExists{footnotehyper.sty}{\usepackage{footnotehyper}}{\usepackage{footnote}}
\makesavenoteenv{longtable}
\usepackage{graphicx}
\makeatletter
\def\maxwidth{\ifdim\Gin@nat@width>\linewidth\linewidth\else\Gin@nat@width\fi}
\def\maxheight{\ifdim\Gin@nat@height>\textheight\textheight\else\Gin@nat@height\fi}
\makeatother
% Scale images if necessary, so that they will not overflow the page
% margins by default, and it is still possible to overwrite the defaults
% using explicit options in \includegraphics[width, height, ...]{}
\setkeys{Gin}{width=\maxwidth,height=\maxheight,keepaspectratio}
% Set default figure placement to htbp
\makeatletter
\def\fps@figure{htbp}
\makeatother
\setlength{\emergencystretch}{3em} % prevent overfull lines
\providecommand{\tightlist}{%
  \setlength{\itemsep}{0pt}\setlength{\parskip}{0pt}}
\setcounter{secnumdepth}{-\maxdimen} % remove section numbering
\ifLuaTeX
  \usepackage{selnolig}  % disable illegal ligatures
\fi
\usepackage[]{natbib}
\bibliographystyle{apalike}

\begin{document}
\maketitle

{
\setcounter{tocdepth}{1}
\tableofcontents
}
\hypertarget{about-the-book}{%
\chapter*{About the Book}\label{about-the-book}}
\addcontentsline{toc}{chapter}{About the Book}

This is the in-progress version of \emph{R Without Statistics}, a forthcoming book from \href{https://www.nostarch.com/}{No Starch Press}.

Since R was invented in 1993, it has become a widely used programming language for statistical analysis. From academia to the tech world and beyond, R is used for a wide range of statistical analysis.

R's ubiquity in the world of statistics leads many to assume that it is only useful to those who do complex statistical work. But as R has grown in popularity, the number of ways it can be used has grown as well. Today, R is used for:

\begin{itemize}
\item
  Data visualization
\item
  Map making
\item
  Sharing results through reports, slides, and websites
\item
  Automating processes
\item
  And much more!
\end{itemize}

The idea that R is only for statistical analysis is outdated and inaccurate. But, without a single book that demonstrates the power of R for non-statistical purposes, this perception persists.

\textbf{Enter R Without Statistics.}

R Without Statistics will show ways that R can be used beyond complex statistical analysis. Readers will learn about a range of uses for R, many of which they have likely never even considered.~

Each chapter will, using a consistent format, cover one novel way of using R.

\begin{enumerate}
\def\labelenumi{\arabic{enumi}.}
\item
  Readers will first be introduced to an R user who has done something novel and learn how using R in this way transformed their work.
\item
  Following this, there will be code samples that demonstrate exactly how the R user did the thing they are being profiled for.
\item
  Finally, there will be a summary, with lessons learned from this novel way of using R.
\end{enumerate}

Written by David Keyes, Founder and CEO of~\href{https://rfortherestofus.com/}{R for the Rest of Us}, R Without Statistics will be published by \href{https://nostarch.com/}{No Starch Press}.

\hypertarget{part-introduction}{%
\part*{Introduction}\label{part-introduction}}
\addcontentsline{toc}{part}{Introduction}

\hypertarget{why-r-without-statistics}{%
\chapter*{Why R Without Statistics?}\label{why-r-without-statistics}}
\addcontentsline{toc}{chapter}{Why R Without Statistics?}

\hypertarget{how-new-zealand-used-r-to-fought-covid-with-r}{%
\section*{How New Zealand Used R to Fought COVID with R}\label{how-new-zealand-used-r-to-fought-covid-with-r}}
\addcontentsline{toc}{section}{How New Zealand Used R to Fought COVID with R}

TODO

\hypertarget{how-i-came-to-use-r}{%
\section*{How I Came to Use R}\label{how-i-came-to-use-r}}
\addcontentsline{toc}{section}{How I Came to Use R}

My own relationship with R goes back to 2016. At the time, I was a consultant, helping non-profits, government agencies, and educational institutions to measure how effective their work is (a field known as \href{https://www.cdc.gov/evaluation/index.htm}{program evaluation}). A lot of my work involved conducting surveys, analyzing the resulting the data, and sharing these results with clients.

The work itself was fine, but the tools I was using to do it were getting on my nerves. Well, one tool really: Excel.

Now look, this is not a place for an anti-Excel rant. Excel is a fine tool that has empowered millions to work with data in ways they would never have been able to without this tool.

But, for me, Excel was tedious. The amount of pointing and clicking I had to do when working with the amount of data I had got old fast. Each time I would conduct a survey, I'd know that it would yield an avalanche of data and that my wrists would end up exhausted after hours of pointing and clicking on Excel buttons.

No matter what I did, analyzing data and creating charts in Excel just involved a lot of repetitive pointing and clicking.

TODO: Add graph of clicks going up

Endless pointing and clicking was just one problem I faced using Excel. Annoying though it was, it didn't affect the quality of my work. Or so I thought until I recalled a project I had worked on a few years earlier.

In this project, I was looking at which school districts in the state of Oregon have \href{https://oregonstate.app.box.com/s/83g5sjdm88xgqdxfze0ri7qo4uff5sj7}{outdoor education programs known as Outdoor School}. As part of this project, I had to download data on all school districts throughout Oregon, filter to only include relevant districts with fifth or sixth graders (the ages Outdoor School takes place), and then merge this with data that I collected as part of a survey I conducted.

I did the work in Excel, using a lot of (you guessed it!) pointing and clicking. The problem came when I was almost done with the project. I've blocked the details from my memory (as I've done with most things Excel-related), but what I do recall is that I wasn't 100\% certain I had done my filtering and joining correctly. And, to make it worse, I had no way to check. Why? Because all my pointing and clicking was ephemeral, gone in the ether as soon as I had completed it.

I finished the Outdoor School project and submitted my report. I think the work I did was probably accurate, but maybe it wasn't?

Now, you may be reading this thinking: why didn't you write down the steps you used in Excel so you could retrace them later? Sure, I could (and should) have done that. But let's be honest: most of us don't.

And the reality is, we're human. We all make mistakes. And without a straightforward way to audit your work (and keeping a list of all of your Excel points and clicks in a separate document is not, in my view, straightforward), mistakes will happen. If you've used Excel to work with data, I guarantee you've made a mistake, just like me.

The good news is that it's ok. There's a solution. And that solution is called R.

\begin{center}\rule{0.5\linewidth}{0.5pt}\end{center}

If I were to redo that project on Outdoor School with R, here's what would be different. Rather than watching points and clicks disappear into the ether, I'd write code that would serve as a record of everything I did. This code would:

\begin{enumerate}
\def\labelenumi{\arabic{enumi}.}
\tightlist
\item
  Download data on all school districts
\item
  Filter to only include districts with fifth or sixth graders
\item
  Join the filtered data on school districts with my survey data
\end{enumerate}

TODO: Add pseudocode to show this in R

Code can be scary. Having to write code is one of the reasons many people never learn R. But code is just a list of things you want to do to your data. It may be written in a hard-to-parse syntax (though it quickly gets easier to make sense of it), but it's just a set of steps. The same steps that we should write out when we're working in Excel, but never do.

If I had done things this way when working on the Outdoor School project, I could have looked back at any point to make sure what I thought was happening to my data was in fact happening. That nagging sensation I had near the end of the project that I may have made a mistake in one of my early points or clicks? It never would come up because I could just review my code to make sure it did what I thought it did.

Using R won't mean you'll never make mistakes again (trust me, you will). But it will mean that you can easily spot your mistakes, make changes, and rerun your code to fix any issues.

I started learning R to avoid tedious pointing and clicking. But what I found was that R improved my work in ways I never expected. It's not just that my wrists are less tired. I now have more confidence in the accuracy of my work.

\hypertarget{my-own-uncertainty-about-the-way-i-use-r}{%
\subsection*{My own uncertainty about the way I use R}\label{my-own-uncertainty-about-the-way-i-use-r}}
\addcontentsline{toc}{subsection}{My own uncertainty about the way I use R}

For the longest time, I felt TK about the way I use R. I use R, a tool designed for statistical analysis, but I don't use it for complex statistical analysis. I don't do machine learning.

\hypertarget{my-background-as-an-anthropologist}{%
\subsection*{My background as an anthropologist}\label{my-background-as-an-anthropologist}}
\addcontentsline{toc}{subsection}{My background as an anthropologist}

\hypertarget{never-used-r-in-grad-school}{%
\subsection*{Never used R in grad school}\label{never-used-r-in-grad-school}}
\addcontentsline{toc}{subsection}{Never used R in grad school}

\hypertarget{i-use-r-for-three-main-things}{%
\subsection*{I use R for three main things:}\label{i-use-r-for-three-main-things}}
\addcontentsline{toc}{subsection}{I use R for three main things:}

\begin{enumerate}
\def\labelenumi{\arabic{enumi}.}
\tightlist
\item
  descriptive stats
\item
  data viz
\item
  RMarkdown
\end{enumerate}

\hypertarget{but-then-i-realized-what-people-get-most-excited-about-is}{%
\subsection*{But then I realized what people get most excited about is:}\label{but-then-i-realized-what-people-get-most-excited-about-is}}
\addcontentsline{toc}{subsection}{But then I realized what people get most excited about is:}

Interview with Sharla (statistician who doesn't do complex stats)
- Her rstudio::conf talk

Data viz (illuminate)

Websites (communicate)

Workflow (i.e.~automate)

Everyone needs to:

\begin{enumerate}
\def\labelenumi{\arabic{enumi}.}
\tightlist
\item
  Illuminate
\item
  Communicate
\item
  Automate
\end{enumerate}

\hypertarget{is-r-just-for-statistics}{%
\subsection*{Is R Just for Statistics?}\label{is-r-just-for-statistics}}
\addcontentsline{toc}{subsection}{Is R Just for Statistics?}

It was a niche language by statisticians for statisticians, now it's used by millions for a huge range of purposes.

\hypertarget{so-if-r-is-a-general-purpose-language-why-use-it-versus-any-other-language}{%
\subsection*{So if R is a general purpose language, why use it versus any other language?}\label{so-if-r-is-a-general-purpose-language-why-use-it-versus-any-other-language}}
\addcontentsline{toc}{subsection}{So if R is a general purpose language, why use it versus any other language?}

Gives you data stuff first, then you can add on everything else

Remove this section? Too defensive?

\hypertarget{book-overview}{%
\section*{Book Overview}\label{book-overview}}
\addcontentsline{toc}{section}{Book Overview}

\hypertarget{how-each-chapter-works}{%
\subsection*{How each chapter works}\label{how-each-chapter-works}}
\addcontentsline{toc}{subsection}{How each chapter works}

\hypertarget{broad-scope-of-book}{%
\subsection*{Broad scope of book}\label{broad-scope-of-book}}
\addcontentsline{toc}{subsection}{Broad scope of book}

\begin{itemize}
\tightlist
\item
  I've tried to choose topics that are relevant to everyone, no matter what you do (e.g.~art with R is cool but not everyone wants to do it)
\end{itemize}

\hypertarget{why-didnt-you-cover-x-topic}{%
\subsection*{Why didn't you cover X topic?}\label{why-didnt-you-cover-x-topic}}
\addcontentsline{toc}{subsection}{Why didn't you cover X topic?}

\begin{itemize}
\tightlist
\item
  That's a great idea, but I can't cover everything!
\item
  The fact that R can do X is a great example of its versatility (please write your own book!)
\end{itemize}

\hypertarget{book-title-is-not-to-be-taken-literally}{%
\subsection*{Book title is not to be taken literally}\label{book-title-is-not-to-be-taken-literally}}
\addcontentsline{toc}{subsection}{Book title is not to be taken literally}

Paul Jarvis Company of One

\hypertarget{part-illuminate}{%
\part*{Illuminate}\label{part-illuminate}}
\addcontentsline{toc}{part}{Illuminate}

\hypertarget{use-general-principles-of-high-quality-data-viz-in-r}{%
\chapter*{Use General Principles of High-Quality Data Viz in R}\label{use-general-principles-of-high-quality-data-viz-in-r}}
\addcontentsline{toc}{chapter}{Use General Principles of High-Quality Data Viz in R}

\hypertarget{develop-a-custom-theme-to-keep-your-data-viz-consistent}{%
\chapter*{Develop a Custom Theme to Keep Your Data Viz Consistent}\label{develop-a-custom-theme-to-keep-your-data-viz-consistent}}
\addcontentsline{toc}{chapter}{Develop a Custom Theme to Keep Your Data Viz Consistent}

\hypertarget{r-is-a-full-fledged-map-making-tool}{%
\chapter*{R is a Full-Fledged Map-Making Tool}\label{r-is-a-full-fledged-map-making-tool}}
\addcontentsline{toc}{chapter}{R is a Full-Fledged Map-Making Tool}

\hypertarget{make-tables-that-look-good-and-share-results-effectively}{%
\chapter*{Make Tables That Look Good and Share Results Effectively}\label{make-tables-that-look-good-and-share-results-effectively}}
\addcontentsline{toc}{chapter}{Make Tables That Look Good and Share Results Effectively}

\hypertarget{part-communicate}{%
\part*{Communicate}\label{part-communicate}}
\addcontentsline{toc}{part}{Communicate}

\hypertarget{use-rmarkdown-to-communicate-accurately-and-efficiently}{%
\chapter*{Use RMarkdown to Communicate Accurately and Efficiently}\label{use-rmarkdown-to-communicate-accurately-and-efficiently}}
\addcontentsline{toc}{chapter}{Use RMarkdown to Communicate Accurately and Efficiently}

\hypertarget{use-rmarkdown-to-instantly-generate-hundreds-of-reports}{%
\chapter*{Use RMarkdown to Instantly Generate Hundreds of Reports}\label{use-rmarkdown-to-instantly-generate-hundreds-of-reports}}
\addcontentsline{toc}{chapter}{Use RMarkdown to Instantly Generate Hundreds of Reports}

\url{https://urban-institute.medium.com/using-r-markdown-to-track-and-publish-state-data-d1291bfa1ec0}

\hypertarget{create-beautiful-presentations-with-rmarkdown}{%
\chapter*{Create Beautiful Presentations with RMarkdown}\label{create-beautiful-presentations-with-rmarkdown}}
\addcontentsline{toc}{chapter}{Create Beautiful Presentations with RMarkdown}

\hypertarget{make-websites-to-share-results-online}{%
\chapter*{Make Websites to Share Results Online}\label{make-websites-to-share-results-online}}
\addcontentsline{toc}{chapter}{Make Websites to Share Results Online}

\hypertarget{part-automate}{%
\part*{Automate}\label{part-automate}}
\addcontentsline{toc}{part}{Automate}

\hypertarget{access-up-to-date-census-data-with-the-tidycensus-package}{%
\chapter*{\texorpdfstring{Access Up to Date Census Data with the \texttt{tidycensus} Package}{Access Up to Date Census Data with the tidycensus Package}}\label{access-up-to-date-census-data-with-the-tidycensus-package}}
\addcontentsline{toc}{chapter}{Access Up to Date Census Data with the \texttt{tidycensus} Package}

\hypertarget{pull-in-survey-results-as-soon-as-they-come-in}{%
\chapter*{Pull in Survey Results as Soon as They Come In}\label{pull-in-survey-results-as-soon-as-they-come-in}}
\addcontentsline{toc}{chapter}{Pull in Survey Results as Soon as They Come In}

\hypertarget{stop-copying-and-pasting-code-by-creating-your-own-functions}{%
\chapter*{Stop Copying and Pasting Code by Creating Your Own Functions}\label{stop-copying-and-pasting-code-by-creating-your-own-functions}}
\addcontentsline{toc}{chapter}{Stop Copying and Pasting Code by Creating Your Own Functions}

\hypertarget{bundle-your-functions-together-in-your-own-r-package}{%
\chapter*{Bundle Your Functions Together in Your Own R Package}\label{bundle-your-functions-together-in-your-own-r-package}}
\addcontentsline{toc}{chapter}{Bundle Your Functions Together in Your Own R Package}

\hypertarget{part-conclusion}{%
\part*{Conclusion}\label{part-conclusion}}
\addcontentsline{toc}{part}{Conclusion}

\hypertarget{come-for-the-data-stay-for-the-community}{%
\chapter*{Come for the Data, Stay for the Community}\label{come-for-the-data-stay-for-the-community}}
\addcontentsline{toc}{chapter}{Come for the Data, Stay for the Community}

\end{document}
